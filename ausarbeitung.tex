\documentclass[a4paper]{scrartcl}

\usepackage[german]{babel}
\usepackage[utf8]{inputenc}
\usepackage[T1]{fontenc}
\usepackage{ae}
\usepackage[bookmarks,bookmarksnumbered]{hyperref}
\usepackage{amsmath}

\begin{document}

\title{Diskreter Logarithmus}
\author{...}
\maketitle

\newpage

\section{Definitionen}
\subsection{Zyklische Gruppe}
\subsection{Prime Restklassengruppe}
\subsubsection{Primitivwurzel}
\subsection{Diskretes Potenzieren}
\subsection{Diskreter Logarithmus}
\section{Algorithmen zur Bestimmung des diskreten Logarithmus}
\subsection{Babystep-Giantstep Algorithmus}
\subsubsection{Theorie}
\subsubsection{Laufzeit}
\subsubsection{Beispiel}
\subsection{Pollard-Rho-Methode}
\subsubsection{Theorie}
\subsubsection{Laufzeit}
\subsubsection{Beispiel}
\section{Anwendungen in der Kryptographie}
\subsection{Elgamal-Verschlüsselungsverfahren}
\subsubsection{Beispiele}
\subsubsection{Schwächen}
\subsection{Anwendungsbeispiel 2}
\section{Ausblick}

\end{document}

