Wie wir mehrmals darauf hingewiesen haben, ist der Diskrete Logarithmus mit derzeitigen Rechenleistungen und Berechnungsverfahren nicht in sinnvoller Zeit berechenbar. \\
In ferner Zukunft könnte sich dies jedoch mit der Einführung schneller und optimierter Quantencomputer ändern. 
In der damit neu entstehenden Quanteninformatik wird dadurch ein Algorithmus berechenbar, der die Quantentheorie benutzt um das diskrete Logarithmusproblem essentiell schneller berechnen kann als jeder bisherige Algorithmus. Dieser Algorithmus wird Shor-Algorithmus genannt. Leider (oder besser gesagt, gut für uns) ist dem Algorithmus derzeit noch eine Technische Hürde auferlegt, da zur Berechnung einer Zahl $n$ man einen Quantencomputer mit $log(n)$ Qubits benötigt.
Neueste Forschungen haben die Qubit-zahl zwar bereits über 1000 gehoben, dies ist jedoch bisher reiner Gegenstand der Forschung. \\
Derzeit werden alle Quantencomputer nur für einen spezifischen Zweck gebaut, laufen nur unter Laborbedingungen und sind noch nicht universell programmierbar. 
Dies wird sich jedoch in Zukunft ändern und spätestens dann muss ein neuer Algorithmus bzw. neue Verfahren gefunden werden, wie trotz oder auch mit Quantencomputern Übertragungen verschlüsselt werden können. 
