
\section{Motivation}
\begin{frame}[c]
Gründe den diskreten Logarithmus näher zu betrachten u.a.:
\begin{itemize}
 \item Grundlage einiger asymmetrischer Kryptosysteme
 \item Alternative zum Faktorisierungsproblem
 \item Basis einiger Schlüsselaustauschprotokolle
\end{itemize}
\end{frame}

\section{Gruppentheorie}

\begin{frame}[c]
Sei $(\mathbb{G}, \odot)$ eine Gruppe, wir definieren:
\begin{definition}
\begin{itemize}
 \item $e$ bezeichnet das neutrale Element
 \item $ord(\mathbb{G}) := |\mathbb{G}|$
 \item Für $\alpha \in \mathbb{G}$ ist $ord(\alpha) = n$ mit $\alpha^n = e$
\end{itemize}
\end{definition}
\end{frame}

\begin{frame}[c]
\begin{definition}
Eine Gruppe $\mathbb{G}$ heißt zyklisch, wenn ein $g \in \mathbb{G}$ existiert, so dass: 
$$
\forall \alpha \in \mathbb{G} : \exists i \in \mathbb{N} : g^i = \alpha
$$
\end{definition}
Wir nennen $g$ einen Generator der zyklischen Gruppe.
\end{frame}

\begin{frame}[c]
\begin{itemize}
 \item Sei $(\mathbb{G}, \odot)$ eine Gruppe und $\alpha \in \mathbb{G}$.
 \item $\alpha $ sei von endlicher Ordnung.
\end{itemize}
\begin{definition}
$ \langle \alpha \rangle $ bezeichnet die von $\alpha$ erzeugte Untergruppe.
\end{definition}
\end{frame}

\begin{frame}[c]
Die Euler'sche $\varphi(n)$"=Funktion ist wie folgt definiert:
\begin{definition}
Sie gibt für eine natürliche Zahl $n$ an, wie viele zu $n$ teilerfremde natürliche Zahlen existieren,
welche nicht größer als $n$ sind:
$$
\varphi(n) \; := \; \Big| \{a \in \mathbb{N} \, |\, 1 \le a \le n \wedge ggT(a,n) = 1 \} \Big|
$$
\end{definition}
\end{frame}

\subsection{Primitivwurzel}
\begin{frame}[c]
Sei $\mathbb{G}$ die prime Restklassengruppe $(\mathbb{Z} /p \mathbb{Z})^\times$
mit der Multiplikation als vorherrschende Operation (gekennzeichnet durch $\times$).
\begin{definition}
Ein Element $\alpha \in \mathbb{G}$ ist eine Primitivwurzel modulo $p$, wenn gilt:
$$
ord(\alpha) = \varphi(p)
$$
\end{definition}
\end{frame}

\subsection{Kleiner fermatscher Satz}
\begin{frame}[c]
\begin{block}{Satz (Fermat)}
Sei $p$ eine Primzahl. Für jede nicht durch $p$ teilbare ganze Zahl $a$ gilt:
\[
 a^{p-1} \equiv 1 \; mod \; p
\]
\end{block}
\end{frame}

\subsection{Diskretes Logarithmus"=Problem}

\begin{frame}[c]
Nachfolgend sei $(\mathbb{G}, \odot)$ eine zyklische Gruppe mit Erzeuger $g$ und $n := ord( \mathbb{G} )$.

\begin{definition}
Die diskrete Exponentialfunktion ist gegeben durch:
\begin{align*}
exp_g: (\mathbb{Z} / n \mathbb{Z}, +) &\to \mathbb{G} \\
k &\mapsto g^k \in \mathbb{G}
\end{align*}
\end{definition}
\end{frame}


\begin{frame}[c]
\begin{definition}
Der diskrete Logarithmus bestimmt das Urbild von $exp_g$:
\begin{align*}
log_g: \mathbb{G} &\to (\mathbb{Z} / n \mathbb{Z}, +) \\
x &\mapsto log_g(x), \; mit \; exp_g(log_g(x)) = x
\end{align*}
\end{definition}

\pause

\begin{exampleblock}{Beispiel}
Sei $\mathbb{G} = (\mathbb{Z} / 13 \mathbb{Z})^\times$ und $g = 2$ eine Primitivwurzel modulo 13:
\begin{table}[H]
\centering
\begin{tabular}{c|c|c|c|c|c|c|c|c|c|c|c|c}
$x$ & 1 & 2 & 3 & 4 & 5 & 6 & 7 & 8 & 9 & 10 & 11 & 12 \\
\hline
$\log_2 \ x$ & 0 & 1 & 4 & 2 & 9 & 5 & 11 & 3 & 8 & 10 & 7 & 6 \\
\end{tabular}
\end{table}
\end{exampleblock}
\end{frame}

\begin{comment}
\begin{frame}[c]
\begin{figure}[htbp]
\centering
\def\svgscale{0.1}
\includesvg{plot}
\end{figure}
\end{frame}
\end{comment}

\subsection{Logarithmusgesetze}

\begin{frame}[c]
Die Logarithmusgesetze gelten auch für den diskreten Logarithmus.

\begin{block}{Satz}
Sei $(\mathbb{G}, \odot)$ eine zyklische Gruppe, $g \in \mathbb{G}, k \in \mathbb{Z}$ und $x, y \in \langle g \rangle$,
dann gilt:
\begin{itemize}
 \item $log_g(x \odot y) = log_g(x) + log_g(y)$
 \item $log_g(x^k) = k \cdot log_g(x)$
\end{itemize}
\end{block}
\end{frame}


\section[Algorithmen]{Algorithmen zur Bestimmung des diskreten Logarithmus}

\subsection{Index"=Calculus}

\begin{frame}[c]
\frametitle{Index"=Calculus}

\begin{tabbing}
\textbf{Eingabe:}~~ \=Eine multiplikative Gruppe $(\mathbb{Z} /p \mathbb{Z})^\times$,\\
\>ein Generator $g$ und ein $\alpha \in (\mathbb{Z} /p \mathbb{Z})^\times$.\\
\textbf{Ausgabe:} \>Der diskrete Logarithmus $x = log_g(\alpha)$
\end{tabbing}

\textbf{Der Algorithmus teil sich in zwei Phasen:}
\begin{enumerate}
 \item Sammeln von Relationen und lösen eines GLS
 \item Berechnung des diskreten Logarithmus für $\alpha$
\end{enumerate}
\end{frame}

\begin{frame}[c]
\textbf{Phase 1:}

\begin{enumerate}
 \item Bestimme die Menge $B = \{ p_1, ..., p_k \}$ aller Primzahlen bis $p_k$ mit $p_k < p$.
 \item Wähle $l \geq k$ zufällige ${x_1, ..., x_l} \in (\mathbb{Z} / (p-1) \mathbb{Z}, +)$ mit:
    \begin{itemize}
     \item $x_i$ und $x_j$ paarweise verschieden für alle $i, j \in \{ 1, ..., l \}$ und
     \item $g^{x_i} \; mod \; p$ ist mit $B$ faktorisierbar
    \end{itemize}
\end{enumerate}
\end{frame}

\begin{frame}[c]
Wir erhalten $l$ Gleichungen:
\begin{align*}
  g^{x_1} &= \displaystyle\prod_{i=1}^{k} p_i^{e_{1,i}} \; mod \; p \\
  & \vdotswithin{=} \\
  g^{x_l} &= \displaystyle\prod_{i=1}^{k} p_i^{e_{l,i}} \; mod \; p
\end{align*}
\end{frame}

\begin{frame}[c]
Logarithmieren liefert uns lineare Gleichungen:
\begin{align*}
  x_1 &= \displaystyle\sum_{i=1}^{k} e_{1,i} \cdot log_g \, p_i \; mod \; (p - 1) \\
  & \vdotswithin{=} \\
  x_l &= \displaystyle\sum_{i=1}^{k} e_{l,i} \cdot log_g \, p_i \; mod \; (p - 1)
\end{align*}
\end{frame}

\begin{frame}[c]
\textbf{Phase 2:}

\begin{enumerate}
 \item Wähle ein zufälliges $x \in (\mathbb{Z} / (p-1) \mathbb{Z}, +)$, sodass:
 $g^{x_i} \; mod \; p$ mit $B$ faktorisierbar ist.
 \item Mit Hilfe der $l + 1$ Gleichungen werden die $log_g \, p_i$ für $i \in \{ 1, ..., k \}$ und $log_g \, \alpha$ bestimmt.
\end{enumerate}
\end{frame}

\begin{frame}[c]
Es gilt:
\begin{align*}
  g^x \cdot \alpha &= \displaystyle\prod_{i=1}^{k} p_i^{e_i} \; mod \; p \\
  \implies log_g \, \alpha &= -x + \displaystyle\sum_{i=1}^{k} e_{i} \cdot log_g \, p_i \; mod \; (p - 1)
\end{align*}
\end{frame}
