
\subsection{Massey"=Omura"=Schema}

\begin{frame}[c]
\frametitle{Massey"=Omura"=Schema}
Überblick:
\begin{itemize}
 \item Das Verfahren stammt von den Kryptologen James Massey und Jim Omura
 \item Es basiert auf dem diskreten Logarithmus"=Problem
 \item Hauptsächlich zum initialen Tausch von Schlüsseln geeignet
\end{itemize}
\bigskip

\pause

\textbf{Besonderheit}:
Es existiert weder ein öffentlicher noch ein gemeinsamer geheimer Schlüssel.
\end{frame}

\begin{frame}[c]
Vorbereitung:
\begin{itemize}
 \item Wir betrachten die prime Restklassengruppe $(\mathbb{Z} / p \mathbb{Z})^\times$
 \item Jeder Teilnehmer $T$ wählt ein beliebiges $e_T$ mit:
 \begin{center}
  $e_T < p - 1$ und $ggT(e_T, p-1) = 1$
 \end{center}
 \item Bestimme $d_T$ (das multiplikative Inverse) mit:
 \begin{center}
 $e_T \cdot d_T \equiv 1 \; mod \; (p - 1)$
 \end{center}
\end{itemize}
\end{frame}

\begin{frame}[c]
Für eine Nachricht $m \in (\mathbb{Z} / p \mathbb{Z})^\times$ gilt nun:
\begin{align*}
(m^{e_T})^{d_T} &= m^{e_T \cdot d_T} \\
&= m^{k \cdot (p-1)+1} \\
&= m^{k \cdot (p-1)} \cdot m \\
&= m \; mod \; p \label{satz:fermat} \tag{*}
\end{align*}

Der Schritt \eqref{satz:fermat} folgt aus dem kleinen Satz von Fermat, da:
\[
m^{k \cdot (p-1)} \equiv 1 \; mod \; p
\]
\end{frame}

\begin{frame}[c]
Ablauf:
\begin{enumerate}
 \item \textbf{Verschlüsselung Alice:} Alice wählt eine Nachricht $m \in (\mathbb{Z} / p \mathbb{Z})^\times$
 und berechnet den Geheimtext:
 \[
  c_{1} = m^{e_A} \; mod \; p
 \]
 \item Alice sendet $c_{1}$ an Bob.
 
 \pause

 \item \textbf{Verschlüsselung Bob:} Bob berechnet den Geheimtext:
 \[
  c_{2} = {c_{1}}^{e_B} \; mod \; p
 \]
 \item Bob sendet $c_{2}$ an Alice.

 \asuivre
\end{enumerate}
\end{frame}

\begin{frame}[c]
\begin{enumerate}
 \suite
 \item \textbf{Entschlüsselung Alice:} Alice hebt mit Hilfe von $d_A$ ihre Verschlüsselung auf:
 \begin{align*}
  c_{3} &= {c_{2}}^{d_A} \; mod \; p \\
  \Leftrightarrow c_{3} &= ( (m^{e_A})^{e_B} )^{d_A} \; mod \; p \\
  \Leftrightarrow c_{3} &= ( (m^{e_A})^{d_A} )^{e_B} \; mod \; p \\
  \Leftrightarrow c_{3} &= m^{e_B} \; mod \; p
 \end{align*}
 \item Alice sendet $c_{3}$ an Bob.

 \asuivre
\end{enumerate}
\end{frame}

\begin{frame}[c]
\begin{enumerate}
 \suite
 \item \textbf{Entschlüsselung Bob:} Bob entschlüsselt die ursprüngliche Nachricht:
 \begin{align*}
  m &= {c_{3}}^{d_B} \; mod \; p \\
  \Leftrightarrow m &= ( m^{e_B} )^{d_B} \; mod \; p
 \end{align*}
\end{enumerate}
\end{frame}

\section{Ausblick}

\begin{frame}[c]
\frametitle{Ausblick}
Peter Shor veröffentlichte 1994 eine Arbeit zum Thema:
\textit{Polynomial-Time Algorithms for Prime Factorization and Discrete Logarithms on a Quantum Computer}
\end{frame}
