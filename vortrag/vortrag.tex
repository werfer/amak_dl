\documentclass{beamer}

\usepackage[german]{babel}
\usepackage[utf8]{inputenc}
\usepackage[T1]{fontenc}
\usepackage{ae}
\usepackage{lmodern}
\usepackage{beamerthemesplit}

\usepackage{graphicx}

\usepackage{amsmath}
\usepackage{verbatim}
%\usepackage{enumitem}
\usepackage{amssymb}
\usepackage{amsthm}
\usepackage{float}
%\usepackage[usenames, dvipsnames]{color}

\usepackage{wrapfig}

\begin{comment}
\usepackage{xcolor}
\hypersetup{
    linktocpage,
    colorlinks,
    linkcolor={green!50!black},
    citecolor={blue!50!black},
    urlcolor={blue!80!black}
}
\end{comment}

\newcommand{\reference}[1]{{[}\small{#1}\normalsize{]}}

\begin{comment}
\theoremstyle{definition}
\newtheorem{definition}{Definition}[section]
\newtheorem{theorem}{Satz}[section]
\newtheorem{lemma}[theorem]{Lemma}
\end{comment}

\usetheme{Berlin}
\usecolortheme{default}
\setbeamertemplate{footline}[frame number]
\setbeamertemplate{itemize items}[square]

\title{Diskreter Logarithmus}
\author{Josef Schmeißer und Fabian Grotz}
\date{25.05.2016}

\begin{document}

\frame{\titlepage}

\frame{\tableofcontents}

\section{Motivation}
\begin{frame}[t]
\end{frame}

\section{Gruppentheorie}

\begin{frame}[c]
Sei $(\mathbb{G}, \cdot)$ eine Gruppe, wir definieren:
\begin{definition}
\begin{itemize}
 \item $e$ bezeichnet das neutrale Element
 \item $ord(\mathbb{G}) := |\mathbb{G}|$
 \item Für $\alpha \in \mathbb{G}$ ist $ord(\alpha) = n$ mit $\alpha^n = e$
\end{itemize}
\end{definition}
\begin{definition}
Sei $n \in \mathbb{N}$ und $\alpha^n = 1$ sowie $\alpha^{n/p} \neq 1$ für alle Primteiler $p$ von $n$.
Dann hat $\alpha$ die Ordnung $n$.
\end{definition}
\end{frame}

\begin{frame}[c]
\begin{definition}
Eine Gruppe $\mathbb{G}$ heißt zyklisch, wenn ein $g \in \mathbb{G}$ existiert, so dass: 
$$
\forall \alpha \in \mathbb{G} : \exists i \in \mathbb{N} : g^i = \alpha
$$
\end{definition}
Wir nennen $g$ einen Generator der zyklischen Gruppe.
\end{frame}

\begin{frame}[c]
\begin{itemize}
 \item Sei $(\mathbb{G}, \cdot)$ eine Gruppe und $\alpha \in \mathbb{G}$.
 \item $\alpha $ sei von endlicher Ordnung.
\end{itemize}
\begin{definition}
$ \langle \alpha \rangle $ bezeichnet die von $\alpha$ erzeugte Untergruppe.
\end{definition}
\end{frame}

\begin{frame}[c]
Die Euler'sche $\varphi(n)$"=Funktion ist wie folgt definiert:
\begin{definition}
Sie gibt für eine natürliche Zahl $n$ an, wie viele zu $n$ teilerfremde natürliche Zahlen existieren,
welche nicht größer als $n$ sind:
$$
\varphi(n) \; := \; \Big| \{a \in \mathbb{N} \, |\, 1 \le a \le n \wedge ggT(a,n) = 1 \} \Big|
$$
\end{definition}
\end{frame}

\begin{comment}
Es sei $(G,\cdot \ p)$ eine Gruppe, $ \alpha \in G $ und $p = |G|$.
$ \alpha $ heißt Primitivwurzel, wenn $ \alpha $ die Ordnung $p$ hat und
$ ggT( \alpha, p) = 1 $.\\
Es sei $ ( Z_p, \cdot \ p ) $ eine prime Restklassengruppe und $ \alpha \in Z_p$.\\
Weiterhin seien $ p_1 , p_2 , \cdots , p_n $ die Primfaktoren von $G$.\\
$ \alpha $ ist eine Primitivwurzel, wenn für $ 1 \leq r \leq n $ gilt $ \alpha^{pr} \neq  1 $.
\end{comment}

\subsection{Primwurzel}
\begin{frame}[c]
Sei $\mathbb{G}$ die prime Restklassengruppe $(\mathbb{Z} /p \mathbb{Z})^\times$
mit der Multiplikation als vorherrschende Operation (gekennzeichnet durch $\times$).
\begin{definition}
Ein Element $\alpha \in \mathbb{G}$ ist eine Primitivwurzel modulo $p$, wenn gilt:
$$
ord(\alpha) = \varphi(p)
$$
\end{definition}
\end{frame}

\end{document}
