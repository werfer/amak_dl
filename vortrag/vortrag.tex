\documentclass{beamer}

\usepackage[german]{babel}
\usepackage[utf8]{inputenc}
\usepackage[T1]{fontenc}
\usepackage{ae}
\usepackage{lmodern}
\usepackage{beamerthemesplit}

\usepackage{graphicx}

\usepackage{amsmath}
\usepackage{verbatim}
%\usepackage{enumitem}
\usepackage{amssymb}
\usepackage{amsthm}
\usepackage{float}
%\usepackage[usenames, dvipsnames]{color}

\usepackage{wrapfig}

\begin{comment}
\usepackage{xcolor}
\hypersetup{
    linktocpage,
    colorlinks,
    linkcolor={green!50!black},
    citecolor={blue!50!black},
    urlcolor={blue!80!black}
}
\end{comment}

\newcommand{\reference}[1]{{[}\small{#1}\normalsize{]}}

\begin{comment}
\theoremstyle{definition}
\newtheorem{definition}{Definition}[section]
\newtheorem{theorem}{Satz}[section]
\newtheorem{lemma}[theorem]{Lemma}
\end{comment}

\usetheme{Berlin}
\usecolortheme{default}
\setbeamertemplate{footline}[frame number]
\setbeamertemplate{itemize items}[square]

\title{Diskreter Logarithmus}
\author{Josef Schmeißer und Fabian Grotz}
\date{25.05.2016}

\begin{document}

\frame{\titlepage}

\frame{\tableofcontents[subsubsectionstyle=hide]}

\section{Motivation}
\begin{frame}[t]
\end{frame}

\section{Gruppentheorie}

\begin{frame}[c]
Sei $(\mathbb{G}, \cdot)$ eine Gruppe, wir definieren:
\begin{definition}
\begin{itemize}
 \item $e$ bezeichnet das neutrale Element
 \item $ord(\mathbb{G}) := |\mathbb{G}|$
 \item Für $\alpha \in \mathbb{G}$ ist $ord(\alpha) = n$ mit $\alpha^n = e$
\end{itemize}
\end{definition}
\begin{definition}
Sei $n \in \mathbb{N}$ und $\alpha^n = 1$ sowie $\alpha^{n/p} \neq 1$ für alle Primteiler $p$ von $n$.
Dann hat $\alpha$ die Ordnung $n$.
\end{definition}
\end{frame}

\begin{frame}[c]
\begin{definition}
Eine Gruppe $\mathbb{G}$ heißt zyklisch, wenn ein $g \in \mathbb{G}$ existiert, so dass: 
$$
\forall \alpha \in \mathbb{G} : \exists i \in \mathbb{N} : g^i = \alpha
$$
\end{definition}
Wir nennen $g$ einen Generator der zyklischen Gruppe.
\end{frame}

\begin{frame}[c]
\begin{itemize}
 \item Sei $(\mathbb{G}, \cdot)$ eine Gruppe und $\alpha \in \mathbb{G}$.
 \item $\alpha $ sei von endlicher Ordnung.
\end{itemize}
\begin{definition}
$ \langle \alpha \rangle $ bezeichnet die von $\alpha$ erzeugte Untergruppe.
\end{definition}
\end{frame}

\begin{frame}[c]
Die Euler'sche $\varphi(n)$"=Funktion ist wie folgt definiert:
\begin{definition}
Sie gibt für eine natürliche Zahl $n$ an, wie viele zu $n$ teilerfremde natürliche Zahlen existieren,
welche nicht größer als $n$ sind:
$$
\varphi(n) \; := \; \Big| \{a \in \mathbb{N} \, |\, 1 \le a \le n \wedge ggT(a,n) = 1 \} \Big|
$$
\end{definition}
\end{frame}

\begin{comment}
Es sei $(G,\cdot \ p)$ eine Gruppe, $ \alpha \in G $ und $p = |G|$.
$ \alpha $ heißt Primitivwurzel, wenn $ \alpha $ die Ordnung $p$ hat und
$ ggT( \alpha, p) = 1 $.\\
Es sei $ ( Z_p, \cdot \ p ) $ eine prime Restklassengruppe und $ \alpha \in Z_p$.\\
Weiterhin seien $ p_1 , p_2 , \cdots , p_n $ die Primfaktoren von $G$.\\
$ \alpha $ ist eine Primitivwurzel, wenn für $ 1 \leq r \leq n $ gilt $ \alpha^{pr} \neq  1 $.
\end{comment}

\subsection{Primwurzel}
\begin{frame}[c]
Sei $\mathbb{G}$ die prime Restklassengruppe $(\mathbb{Z} /p \mathbb{Z})^\times$
mit der Multiplikation als vorherrschende Operation (gekennzeichnet durch $\times$).
\begin{definition}
Ein Element $\alpha \in \mathbb{G}$ ist eine Primitivwurzel modulo $p$, wenn gilt:
$$
ord(\alpha) = \varphi(p)
$$
\end{definition}
\end{frame}


\section{Algorithmen zur Bestimmung des diskreten Logarithmus}

\subsection{Index-Calculus}

% to do %

\subsection{Babystep-Giantstep-Algorithmus}
\subsubsection{Theorie}
\begin{frame}
	\frametitle{Babystep-Giantstep-Algorithmus}
	\framesubtitle{Theorie}
	\begin{itemize}
		\item Gegeben Zyklische Gruppe $\mathbb{G}$ mit Ordnung $n$, Generator $g$ und ein Element der Gruppe $\alpha$ 
		\item Gesucht ist $x$ sodass $g^x = \alpha$
		\item setzen $x = i\cdot m + j$
		\item $m$ sollte in $\lceil \sqrt{n} \  \rceil$ sein
		\item ausserdem $ 0 \leq i < m $ und $ 0 \leq j < m$
	\end{itemize}
\end{frame}

\begin{frame}
	\frametitle{Theorie}
	\textbf{$g^{im+j} = \alpha \Leftrightarrow  g^j = a(g^{-m})^i$} \pause 
	\begin{block}{Babystep}
		Berechne für alle $j$ den Ausdruck $g^j$.
		Paare $(j,g^j)$ werden in Tabelle gespeichert
	\end{block} \pause
	\begin{block}{Giantstep}
		Berechne $(g^{-m})^i$ und vergleiche mit Tabelle.
		Wenn Treffer, gib $x = im + j$ aus. 
	\end{block}
\end{frame}

\subsubsection{Algorithmus in Pseudocode}
\begin{frame}
	\frametitle{Algorithmus in Pseudocode}
	
	\begin{block}{Eingabe}
		zyklische Gruppe $G$ der Ordnung $n$ mit einem Generator $g$ und ein Element der Gruppe $\alpha$
	\end{block}
	\begin{block}{Berechnung}
		\begin{enumerate}
			\item Setze $m:=\lceil\sqrt{n}\rceil$ 
			\item Für alle $j \in \{0,\cdots,m-1\}$:
			\begin{enumerate}
				\item Berechne $g^j$ und speichere das Tupel $(j,g^j)$ in einer Tabelle  
			\end{enumerate} 
			\item Setze $t:=\alpha$
			\item Für alle $i \in \{ 0,\cdots, m-1\}$:
			\begin{enumerate}
				\item Suche in der Tabelle nach einem Paar mit $t=g^j$
				\item Wenn Paar existiert gib $im+j = log_g(\alpha)$  aus
				\item Wenn nicht: Setze $t := t \ast g^{-m}$ und fahre fort 
			\end{enumerate} 
		\end{enumerate} 
	\end{block}
\end{frame}

\subsubsection{Beispiel}
\begin{frame}
	\frametitle{Beispiel}
	\begin{itemize}
		\item Wir nehmen eine Gruppe $ G $ der Ordnung $n = 29$ mit Erzeuger $g = 11$ 
		\item Wir wollen den diskreten Logarithmus von $a = 3$ zur Basis g berechnen, also die Lösung von $3 = 11^x \bmod 29$
		\item Rechnung siehe Tafel \pause
		\item Lösung:   $log_{11} 3 = 2 \cdot 6 + 5 = 17$
	\end{itemize} 
\end{frame}

\subsubsection{Laufzeit}
\begin{frame}
	\frametitle{Laufzeit}
	\begin{block}{Laufzeit}
		hängt von Liste ab, die m-Einträge nach Babystep Berechnungen hat und durchsucht werden muss.\\ Mit Hashfunktionen kann laufzeit gemindert werden.\\
		Dadurch gesamte Laufzeit bei $O(m) bzw. O(\sqrt{n})$
	\end{block}
\end{frame}

\subsubsection{Speicherverbrauch}
\begin{frame}
	\frametitle{Speicherverbrauch}
	\begin{block}{Speicherverbrauch}
		hängt von Liste ab, die m-Einträge nach Babystep Berechnungen hat.
		Dadurch gesamter Speicherbedarf bei $O(m) bzw. O( \sqrt{n}) $
	\end{block}
\end{frame}

\section{Anwendungen in der Kryptographie}

\subsection{Elgamal-Verschlüsselungsverfahren}
\begin{frame}
	\frametitle{Elgamal-Verschlüsselungsverfahren}
	\begin{block}{Elgamal-Verschlüsselungsverfahren}
		\begin{itemize}
			\item entwickelt 1985 von Taher Elgamal
			\item ist ein Public-Key Verschlüsselungsverfahren
			\item beruht auf Operationen in einer zyklischen Gruppe endlicher Ordnung
		\end{itemize}
	\end{block}
\end{frame}
\subsubsection{Vorbereitungen}
\begin{frame}
	\frametitle{Vorbereitungen}
	Der Empfänger
	\begin{enumerate}
		\item wählt eine endliche, zyklische Gruppe $G$ der Ordnung $n$ mit Erzeuger $p$ 
		\item wählt eine zufällige Zahl $a \in \{1,\ldots,n-1\} \ mit \ ggT(a,n)=1$ als privater Schlüssel des Empfängers
		\item berechnet das Gruppenelement $A = p^a \ \in \ G$ als öffentlicher Schlüssel 
		\item veröffentlicht $(G,p)$ und $A$ 
	\end{enumerate}
\end{frame}

\subsubsection{Verschlüsseln}
\begin{frame}
	\frametitle{Verschlüsseln}
	Der Sender
	\begin{enumerate}
		\item möchte Nachricht $m \in G$ versenden 
		\item wählt $r \in \{1,\ldots,n-1 \} \ mit \ ggT(r,n) = 1$ 
		\item berechnet $R = p^r \in G$  
		\item berechnet $c = A^r \cdot n \in G $ 
		\item sendet (R,c) an den Empfänger
	\end{enumerate}
\end{frame}

\subsubsection{Entschlüsseln}
\begin{frame}
	\frametitle{Entschlüsseln}
	Der Empfänger
	\begin{itemize}
		\item berechnet  $ m = \ R^{-a} \cdot c  \in G$ 
	\end{itemize}\pause 
	Es gilt: $R^{-a} \cdot c = p^{-ra} \cdot A^r \cdot m  = p^{-ra} \cdot p^{ar} \cdot m = m$
\end{frame}

\subsubsection{Beispiel}
\begin{frame}
	\frametitle{Beispiel}
	Wie nehmen ein Beispiel:\\
	$p = 47,\ \ \ g = 5 $   werden veröffentlicht\\
	B wählt $b=29$\\
	A wählt $a=7$\\
	Nachricht $m = 42$\\
\end{frame}

\subsubsection{Decisional Diffie-Hellman-Problem}
\begin{frame}
	\frametitle{Decisional Diffie-Hellman-Problem (DDH)}
	\begin{block}{Grundgedanke}
		Angreifer kann zwischen $ \langle g^a , g^b, g^{ab} \rangle $ und $ \langle g^a, g^b, g^c \rangle $ nicht unterscheiden, wenn $a,b$ und $c$ zufällig gewählt in $[1,\lvert G\rvert]$. 
	\end{block}
\end{frame}


\subsection{Massey-Omura-Schema}

%to do%

\end{document}
